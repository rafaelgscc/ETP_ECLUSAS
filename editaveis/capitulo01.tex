\part{Considerações Iniciais}

\chapter[Introdução]{Introdução}
 
As eclusas desempenham um papel fundamental na infraestrutura de transporte aquaviário, facilitando a navegação de embarcações em rios, canais e vias navegáveis interiores. Desde os tempos antigos, as eclusas têm sido utilizadas para superar diferenças de nível em corpos d'água, permitindo que embarcações naveguem em trechos de água com diferentes altitudes. Ao longo dos séculos, as eclusas evoluíram de estruturas rudimentares para sofisticados sistemas de engenharia que desempenham um papel vital no comércio e no transporte de mercadorias em todo o mundo.

O funcionamento das eclusas é baseado no princípio da elevação ou descida controlada das embarcações entre diferentes níveis de água. As eclusas consistem em câmaras de água fechadas por comportas, onde as embarcações entram e são então elevadas ou baixadas conforme necessidade de operação em determinados horários. Esse processo é controlado por meio de sistemas hidráulicos e eletromecânicos, permitindo que as embarcações naveguem em trechos de água com com alturas diferentes tendo como referência o nível do mar.

Sob a responsabilidade do Departamento e Infraestrutura de Transportes (DNIT) existem oito eclusas. Quatro delas ficam no Rio grande o sul (Amarópolis, Bom Retiro do Sul, Dom Marco e Fandango) duas em São Paulo (Jupiá e Três Irmãos), uma no Pará (Tucuruí) e uma na Bahia (Sobradinho).

Este trabalho tem como objetivo da criação de um sistema de controle do supervisório da eclusa de sobradinho sobradinho, usando equipamentos de hardware aberto e com conceito de Internet das Coisas (IoT). Para a criação do software do supervisório será com ferramentas disponíveis no DNIT, com o objetivo de visualização em tempo real do status da eclusa.

\section{Objetivos Gerais}

Este trabalho tem como objetivo investigar os princípios de engenharia eletrônica aplicados na automação de eclusas, analisando os sistemas de controle, sensores, atuadores e protocolos de comunicação utilizados. 


\section{Objetivos Específicos}

%Investigar as tecnologias de automação empregadas em eclusas, com foco na engenharia eletrônica e utilizando soluções de hardware aberto, a fim de Explorar as possibilidades de aplicação de soluções de hardware aberto, como Arduino, Raspberry Pi ou outras plataformas similares, para controle e monitoramento das operações das eclusas. Desenvolver e implementar um sistema de automação de eclusas utilizando soluções de hardware aberto, incluindo a seleção e integração de sensores, atuadores e outros componentes eletrônicos necessários. Avaliar a viabilidade técnica e operacional da utilização de soluções de hardware aberto em comparação com sistemas tradicionais de automação baseados em CLPs, considerando critérios como desempenho, custo, flexibilidade e escalabilidade.
%Investigar os desafios específicos relacionados à implementação de soluções de hardware aberto em ambientes críticos, como eclusas, incluindo aspectos de confiabilidade, segurança cibernética e interoperabilidade com sistemas legados. Avaliar os benefícios e limitações das soluções de hardware aberto em termos de impacto econômico, destacando oportunidades de redução de custos, aumento da acessibilidade tecnológica e potencial para inovação a  e Criar um protótipo com hardware aberto da eclusa de sobradinho, mostrando seu funcionamento como paradigma.

\begin{enumerate}
    \item Investigar as tecnologias de automação empregadas em eclusas, com foco na engenharia eletrônica e utilizando soluções de hardware aberto, a fim de:
    \begin{itemize}
        \item Explorar as possibilidades de aplicação de soluções de hardware aberto, como Arduino, \textit{Raspberry Pi}, ESP32 ou outras plataformas similares, para controle e monitoramento das operações das eclusas.
        \item Desenvolver e implementar um sistema de automação de eclusas utilizando soluções de hardware aberto, incluindo a seleção e integração de sensores, atuadores e outros componentes eletrônicos necessários.
        \item Avaliar a viabilidade técnica e operacional da utilização de soluções de hardware aberto em comparação com sistemas tradicionais de automação baseados em CLPs, considerando critérios como desempenho, custo, flexibilidade e escalabilidade.
    \end{itemize}
    
    \item Investigar os desafios específicos relacionados à implementação de soluções de hardware aberto em ambientes críticos, como eclusas, incluindo aspectos de confiabilidade, segurança cibernética e interoperabilidade com sistemas legados.
    
    \item Avaliar os benefícios e limitações das soluções de hardware aberto em termos de impacto econômico, destacando oportunidades de redução de custos, aumento da acessibilidade tecnológica e potencial para inovação.
    
    \item Criar um protótipo da usabilidade de um supervisório, mostrando seu funcionamento como paradigma.
\end{enumerate}


\subsection{Justificativa}

Com a necessidade de desenvolver um modelo de controle que seja sustentável para o DNIT, se criou a necessidade de um sistema que fosse feito \textit{in house} com robustês e economia. Para com isso, criar um modelo que seja replicado para outras eclusas no Brasil. 




\section{Estrutura do trabalho}

A partir dos objetivos gerais e específicos, delimitamos o escopo do trabalho para que possamos mostrar como será para o TCC2 a estrutura de trabalho. O trabalho é divido em 4 partes: $I)$ Considerações Iniciais $II)$  Referncial teórico $III)$ Decisões de projeto e trabalho e $IV)$ Conclusões e trabalhos futuros.



No capítulo 1 será dada uma introdução e os objetivos do trabalho;

No capítulo 2 será abordado o referencial teórico do trabalho;

No capítulo 3 Será mostrado decisões de engenharia acerca do projeto proposto e as escolhas dos componentes para controle e software de interface;

No capítulo 4 Será abordado a estrutura de prototipagem, e das ferramentas de hardware aberto;

No capítulo 5 Será a implementação do projeto no software de supervisório e os itens para implementação do projeto na eclusa para os operadores;

No capítulo 6  Será dada as conclusões sobre o trabalho e os trabalhos futuros.

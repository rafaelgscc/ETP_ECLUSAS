\begin{resumo}
Eclusas são sistemas que trabalham como elevadores para a movimentação de embarcações em ambos os sentidos de um rio. Para seu melhor controle, um sistema supervisório é imprescindível para o monitoramento de toda a planta. Neste contexto, o uso de um protocolo industrial como o MODBUS, o uso do controlador ESP32 e de um supervisório de código aberto SCADABR são bastantes proveitosos, pois além de serem amplamente usados em industrias, possuem uma característica de fácil manipulação e controle por parte dos operadores e engenheiros. Neste trabalho será abordado a implementação de um sistema supervisório em uma eclusa, e toda a eletrônica por trás da implementação, bem como um protótipo para teste de paradigma. Na primeira parte, serão tratados conceitos a cerca do que é uma eclusa, e tecnologias que a compõem. Logo após serão discutidos tecnologias a serem implementadas, resultados e recomendações para os engenheiros e operadores das eclusas. 

 \vspace{\onelineskip}
    
 \noindent
 \textbf{Palavras-chaves}: Eclusas. \textit{Modbus}. \textit{SCADA}. ESP32.
\end{resumo}
